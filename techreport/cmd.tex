\section{Available Commands}
\label{sec.cmd}

This section presents the list of commands offered by the $\pvslm$
tool. Section~\ref{sec.nasalib} presents examples on the use of some
of these commands.

The $\pvslm$ tool provides commands at two levels. First, it provides
commands at the level of library sources for managing library $\git$
repositories.  Second, it provides commands at the level of packages
for managing the contents of a library. Table~\ref{tab.cmd} lists all
commands $\pvslm$ offers.

At the level of library sources, the tool provides $5$ different
commands, all identified with the special token \cde{src}. They
include commands for:

\begin{itemize}
  \item Adding a library source (i.e., command \cde{-a}) with a
    name, a short description, and a $\git$ URL.
  \item Deleting a library source (i.e., command \cde{-d}) with
    the given name.
  \item Cloning a library source (i.e., command \cde{-c}) with the
    given name.
  \item Updating a library source (i.e., command \cde{-u}) with
    the given name.
  \item Removing the clone of a library source (i.e., command
    \cde{-r}) with the given name.
\end{itemize}

It is important to note that none of the commands at the library
source level modify the user's PVS installation. These commands
exclusively modify the internals of the $\pvslm$ configuration. Also,
note that a library source is realized exactly by one $\git$
repository.

At the level of packages, the tool provides $4$ different
commands, all identified with the special token \cde{pkg}. They
include commands for:

\begin{itemize}
  \item Installing and updating a given package (i.e., command
    \cde{-i}) from a given library source, including {\em all} its
    dependencies.
  \item Updating a given package (i.e., command \cde{-u}) from a given
    library source, including {\em all} its dependencies.
  \item Deleting a given package (i.e., command \cde{-d}) from a given
    library source (local copy), including all packages that depend on
    it.
  \item Listing the contents (i.e., command \cde{-l}) from a given
    library source.
\end{itemize}

The listing command has three variants. In the first one, all
libraries available to the system are listed. In the second one, all
packages of a given library are listed. In the third one, all
dependencies of a given package and library are listed. Internally,
the $\pvslm$ uses a topological sorting algorithm, based upon each
package's metadata, for computing the set of dependencies among
theories and packages.

\begin{table*}[pthb]
  \begin{center}
    \begin{tabular}{ | l | l | l | p{6cm} | }
      \hline Level & Command & Parameters & Description \\ \hline
      src & -a          & name desc URL         & Add a new library source with the given name, description, and URL.                   \\ \cline{2-4}
          & -d          & name                  & Delete the given library source.                                          \\ \cline{2-4}
          & -c          & name                  & Clone the given library source.                          \\ \cline{2-4}
          & -u          & name                  & Update the given library source.                                \\ \cline{2-4}
          & -r          & name                  & Remove the clone of the given library source.                                                  \\ \hline
      pkg & -i          & library@package       & Install and update the given package, and all its dependencies.                \\ \cline{2-4}
          & -u          & library@package       & Update a package and all its dependencies.                   \\ \cline{2-4}
          & -d          & library@package       & Delete a  package and all ones depending on it.           \\ \cline{2-4}
          & -l          &                       & List the installed libraries.                                       \\ \cline{2-4}
          & -l          & library               & List the available packages of the given library.                         \\ \cline{2-4}
          & -l          & library@package       & List all the dependencies of the given package.                               \\ \hline
    \end{tabular}
  \end{center}
  \caption{$\pvslm$ command list.}
  \label{tab.cmd}
\end{table*}

