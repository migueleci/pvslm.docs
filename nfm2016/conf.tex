\section{The $\pvslm$ Description Language}
\label{sec.conf}

This section presents a formal definition of the $\pvslm$ description
language in the form of BNF-like notation. It also presents the main
conventions and assumptions used by the $\pvslm$ tool for managing PVS
libraries.


\paragraph{Terminology.}
The $\pvslm$ tool distinguishes three levels of aggregation for PVS
sources, namely, theories, packages, and libraries. A PVS theory is
the building block of a source. A {\em package} is a collection of
theories.  A {\em library} is at the top level of aggregation,
comprising a collection of packages. In summary, a library is a
collection of packages and a package is a collection of theories. The
$\pvslm$ tool can manage several libraries each with several packages.

\paragraph{Package configuration.}
A package is defined in a folder at the root of the library source,
with the package inheriting its name from the folder's name. Each
folder contains the \cde{pvs} and \cde{prf} files of its theories.  It
also includes a folder named \cde{pvsbin}: this is a special folder
used by the $\pvslm$ implementation for accessing the {\em package
  metadata} from a file named \cde{top.dep}. A package metadata is
defined by using $\pvslm$'s description language,
presented in BNF-like notation in Table~\ref{tab.bnf}.

\begin{table}
  \centering
  \begin{tabular}{r c p{8cm}}
    \hline \\
    $\nterm{\cde{metadata}}$ & ::= & $\nterm{\cde{header}} \; \nterm{\cde{body}}$ \\
    $\nterm{\cde{header}}$ & ::= & `/' $\nterm{\cde{theorylist}}$ \\
    $\nterm{\cde{theorylist}}$ & ::= & $\nterm{\cde{theory}} \mid \nterm{\cde{theory}}$ `,' $\nterm{\cde{theorylist}}$ \\
    $\nterm{\cde{body}}$ & ::= & $(\nterm{\cde{packagedep}} \mid \nterm{\cde{theorydep}})*$ \\
    $\nterm{\cde{packagedep}}$ & ::= & $\nterm{\cde{package}}$ `/' $\nterm{\cde{theorylist}}$ \\
    $\nterm{\cde{theorydep}}$ & ::= & $\nterm{\cde{theory}}$ `:' $\nterm{\cde{qualtheorylist}} ?$ \\
    $\nterm{\cde{qualtheorylist}}$ & ::= & $\nterm{\cde{qualtheory}} \mid \nterm{\cde{qualtheory}}$ `,' $\nterm{\cde{qualtheorylist}}$ \\
    $\nterm{\cde{qualtheory}}$ & ::= & $(\nterm{\cde{package}}\; \text{`@'})?$ $\nterm{\cde{theory}}$ \\
    \\
    \hline
  \end{tabular}
  \caption{Syntax of the $\pvslm$ description language.}
  \label{tab.bnf}
\end{table}

The topmost symbol in the $\pvslm$ description language is
$\nterm{\cde{metadata}}$, while $\nterm{\cde{theory}}$ and
$\nterm{\cde{package}}$ are terminals representing, respectively,
theory and package names. The metadata comprises two parts, namely, a
header and a body. The header corresponds to a single line with the
`/' character followed by a comma-separated list of theory names;
these names correspond to the names of the theories included in the
package (in any order). A body comprises any number of lines, each
with either a package dependency or a theory dependency. A package
dependency describes a dependency from another package and the list of
theories from that package that are being depended upon. A theory
dependency describes, for each one of the theories listed in the
header of the package, the list of its theory dependencies. In the
case such a theory depends on a theory from another package, the name
of that dependency must be qualified by the name of its package.

\begin{figure}
  \centering
  \includegraphics[width=11cm]{images/top.png}
  \caption{Overview of metadata file \cde{top.dep} for package
    \cde{trig} in NASALib.}
  \label{fig.top}
\end{figure}

Figure~\ref{fig.top} presents an overview of the configuration file
for package \cde{trig} in NASALib. According to its description
header, package \cde{trig} defines theories \cde{top},
\cde{trig\_doc}, \cde{trig}, \cde{trig\_values}, etc. It contains $5$
package dependencies and $6$ theory dependencies. For instance,
package \cde{trig} depends on theory \cde{for\_iterate} in package
\cde{structures}, and on theories \cde{finite\_sets\_minmax} and
\cde{finite\_sets\_inductions} in package \cde{finite\_sets}. On the
other hand, theory \cde{trig\_doc} has no dependencies, while theory
\cde{trig} depends on theories \cde{trig\_basic}, \cde{sqrt},
\cde{trig\_values}, and \cde{trig\_ineq}. In the case of theory
\cde{sqrt}, it is explicitly stated that such a theory is available
from package \cde{reals}.

