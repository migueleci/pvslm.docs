\section{Case study: Managing NASALib with $\pvslm$}
\label{sec.nasalib}

This section presents a step-by-step guide on the configuration and
installation of the NASA PVS Library (NASALib) with the help of the
$\pvslm$ implementation. The current release of NASALib is annotated
with $\pvslm$'s description language and it is freely available from a
$\git$ repository in GitHub.

\paragraph{Library source creation.}
The first step is to issue a command for configuring the library sources
as follows:
%
{\small\begin{verbatim}
$ pvslm.py src -a \\
   nasalib \\
   `The NASA PVS Library is a collection of formal PVS developments \\
    maintained by the NASA Langley Formal Methods Team.' \\
   https://github.com/nasa/pvslib.git
\end{verbatim}}
%
\noindent This command generates the library source configuration file
in the $\pvslm$ installation folder with the given information: a
library named \cde{nasalib}, with the given description, and whose
packages are available as a $\git$ repository from the given URL. At
this point, the configuration file is created but no packages are
downloaded nor installed from the repository. Note that it is not
possible to have two or more library sources with the same name.

It is important to note that this step is unnecessary with the current
distribution of $\pvslm$ because its installation script automatically
creates a library source configuration file for \cde{nasalib}. This
step is included for the sake of completeness of the example.

\paragraph{Downloading a library.} It is possible to use the $\pvslm$
implementation for downloading a library once its library source file
has been created. The following command downloads NASALib:
%
\begin{verbatim}
$ pvslm.py src -c nasalib
\end{verbatim}
%
Internally, this command uses $\git$'s cloning technology so that the
NASALib repository is copied locally from the URL configured with the
library source.

\paragraph{Updating a library.} It is also possible to update a local
copy of a library. For updating the local copy of NASALib, an user can
issue the following command:
%
\begin{verbatim}
$ pvslm.py src -u nasalib
\end{verbatim}
%
The effect of this command is to update the entire local copy of
\cde{nasalib} via $\git$'s pull command from the URL configured with
the library source.

\paragraph{Listing the contents of a library.} Once a local
copy of a library is available, it is possible to list all its
packages and their dependencies. The following command can be issued
to list the dependencies of package \cde{complex} in NASALib:
%
\begin{verbatim}
$ pvslm.py pkg -l nasalib@complex
\end{verbatim}
%
The output of this command is as follows:
%
{\small\begin{verbatim}
Package complex depends on:
algebra
analysis_ax
ints
lnexp
reals
structures
trig
\end{verbatim}}

\paragraph{Installing a package in PVS.} The $\pvslm$ implementation
can install a package from a local copy of a library in a working PVS
environment. The following command installs package \cde{complex} from
NASALib:
%
\begin{verbatim}
$ pvslm.py pkg -i nasalib@complex
\end{verbatim}
%
Since package \cde{complex} depends on other packages, the $\pvslm$
tool will ask for permission to install all these dependencies if they
are not already installed. The following is the output to the user for
this installation command.
%
{\small\begin{verbatim}
Package complex depends on:
...
Would you like to install the package(s) (y/N): 
\end{verbatim}}
%
By convention, $\pvslm$ performs {\em all} installations in the folder
specified by the \cde{PVS\_PATH} global variable.

\paragraph{Deleting a package.} Finally, a user can issue the following
command to delete package \cde{ints} from the (internal) local copy of
NASALib:
%
\begin{verbatim}
$ pvslm.py pkg -d nasalib@ints
\end{verbatim}
%
If there are packages that depend on the package to be removed, the
$\pvslm$ tool will list all of them and will ask the user for
authorization to also remove them.
