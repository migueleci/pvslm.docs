\section{Introduction}
\label{sec.intro}


The Prototype Verification System~\cite{pvs-cade92} (PVS) is a
verification system comprising a specification language integrated
with support tools and a theorem prover; basically, PVS is a
mechanization of classical typed higher-order logic with
specifications organized into parameterized theories in this logic.
The PVS system has been used in state-of-the-art formal methods
projects -- both in industry and in research -- as a productive
environment for constructing and maintaining collections of
theories. These novel developments often result in large collections
of PVS definitions, theorems, and proofs, grouped into libraries, with
many dependencies among them. This is the case, for instance, of the
NASA PVS Library~\cite{nasalib} maintained by the NASA Langley Formal
Methods Team: a large collection of formal developments in PVS
comprising more than 40 theories, ranging from trigonometry to graphs
and topology.




that is
freely available to the formal methods community.



that include novel developments in the form of mechanized theories.



a
mechanized environment for formal specification and verification.  PVS
consists of a specification language, a number of predefined theories,
a type checker, an interactive theorem prover that supports the use of
several decision procedures and a symbolic model checker.  Nowadays,
PVS is being used as a productive environment for constructing and
maintaining large formalizations and proofs. PVS is extensible via
libraries, collections of theories comprising formal developments that
include definitions, theorems, and proofs in the PVS language. Given
the increasing size of mathematical developments in the form of
libraries, and the growing need for and adoption of mechanized proof
environments~\cite{avigad-mech14,hales-proofs14}, it is important to
have tools for managing these libraries. This paper presents
$\pvslm$~\cite{pvslm}, an open source tool for managing PVS libraries
that is freely available for download~\cite{pvslm} under the GNU
General Public License GPLv3.

The $\pvslm$ tool features support for different library sources,
libraries with several theories, and dependencies among theories
within the same library. It uses a small footprint description
language to annotate libraries so they can be shared among PVS users
with the help of the $\pvslm$ tool. This language is intended to
document each library with the information of theories it offers and
the dependencies among them. This paper presents the description
language in the form of BNF-like notation; its use is illustrated with
actual running examples.

The $\pvslm$ tool is a command line tool written in the Python
programming language. It can manage {\em any} (annotated) library that
is publicly available from a $\git$ server through the internet.  Once
available, such a library source can be configured in $\pvslm$, be
automatically downloaded from the internet and set up in the host
system. For internet downloads, the $\pvslm$ tool depends on {\em
  curl}, a command line tool commonly found in Linux and Mac OS X
installations. The $\pvslm$ tool features support for updating,
deleting, and re-installing the contents of a library source.  Several
library sources can be configured in $\pvslm$.

The current distribution of $\pvslm$ automatically installs the latest
version of the NASA PVS Library~\cite{nasalib}, a collection of formal
PVS developments maintained by the NASA Langley Formal Methods
Team. As a case study for the use of the $\pvslm$ tool, this paper
presents a step-by-step guide on how to manually configure the NASA
PVS Library comprising the command line instructions and snapshots of
the interaction.

\paragraph{Paper outline.} Section~\ref{sec.pvslm} presents the relevance 
of the tool for PVS users. Section~\ref{sec.conf} presents the
annotation language for configuring a PVS library and introduces some
assumptions on the internal structure of the library. Section~\ref{sec.install} 
presents an overview of the tool's installation. Section~\ref{sec.nasalib} 
presents a step-by-step guide on the installation and management of 
the NASA PVS Library that is available from GitHub. Finally, 
Section~\ref{sec.concl} presents some concluding remarks.
