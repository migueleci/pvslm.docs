\section{Introduction}
\label{sec.intro}

The Prototype Verification System~\cite{pvs-cade92} (PVS) is a
mechanized environment for formal specification and verification.  PVS
consists of a specification language, a number of predefined theories,
a type checker, an interactive theorem prover that supports the use of
several decision procedures and a symbolic model checker.  Nowadays,
PVS is being used as a productive environment for constructing and
maintaining large formalizations and proofs. PVS is extensible via
libraries, collections theories comprising formal developments in PVS
that include definitions, theorems, and proofs. Given the increasing
size of mathematical developments in the form of libraries and the
growing adoption of mechanized proof environments for checking formal
results~\cite{}, it is important to have tools for managing
libraries. This paper presents $\pvslm$~\cite{pvslm}, an open source
tool for managing PVS libraries that is freely available for
download~\cite{pvslm} under the GNU General Public License GPLv3

The $\pvslm$ tool features support for different library sources,
libraries with several theories, and dependencies among libraries and
theories (even from different library sources). It uses a small
footprint description language to annotate libraries so they can be
shared among PVS users through the $\pvslm$ tool. This language is
intended to document each library with the information of theories it
offers and the dependencies among them, including those dependencies
from other libraries. This paper presents the description language in
the form of BNF-like notation and its use is illustrated with running
examples.

The $\pvslm$ tool is a command line tool written in the Python
programming language. It can manage {\em any} (annotated) library that
is publicly available from a $\git$ server through the internet.
Once available, such a library source can be configured in $\pvslm$,
and be automatically downloaded from the internet and set up in the
host system. For internet downloads, the $\pvslm$ tool depends on {\em
  curl}, a command line tool commonly found in Linux and Mac OS X
installations. The $\pvslm$ tool features support for updating,
deleting, and re-installing the contents of a library source.  Several
library sources can be configured in $\pvslm$, with the tool managing
automatically dependencies among them.

The distribution of $\pvslm$ automatically installs the latest version
of the NASA PVS Library~\cite{nasalib}, a collection of formal PVS
developments maintained by the NASA Langley Formal Methods Team. As a
case study for the use of the $\pvslm$ tool, this paper presents a
step-by-step guide on how to manually configure the NASA PVS Library
comprising the command line instructions and snapshots of the
interaction.

\paragraph{Paper outline.} Section~\ref{sec.conf} presents the
annotation language for configuring a PVS library and introduces some
assumptions on the internal structure of the library.
Section~\ref{sec.install} presents an overview of the tool's installation
process, while Section~\ref{sec.arch} presents the architecture of the tool.
Section~\ref{sec.cmd} includes the list of commands available from $\pvslm$.
Section~\ref{sec.nasalib} presents a step-by-step guide on the
installation and management of the NASA PVS Library that is available
from GitHub. Finally, Section~\ref{sec.concl} presents some concluding
remarks.
