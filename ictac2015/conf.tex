\section{Library Configuration}
\label{sec.conf}

This section describes the conventions and assumptions made by
$\pvslm$ for managing PVS sources.

\paragraph{Conventions.}
The $\pvslm$ tool distinguishes three levels of aggregation for PVS
sources. A PVS theory is the building block of a library managed by
$\pvslm$. A {\em package} is a collection of theories. At the top
level of aggregation is a {\em library}, which comprises a collection
of packages. Namely, a library is a collection of packages and a
package is a collection of libraries. The $\pvslm$ tool can manage
several libraries each with several packages. However, the tool does
not support management at the level of specific theory files.

\paragraph{Package configuration.}
Each package of a library is defined in a folder at the root of the
library source. The folder's name corresponds to the name of the
package. Each folder contains the corresponding \cde{pvs} and
\cde{prf} files for its theories, and the folder \cde{pvsbin}, a
special folder with the package's metadata described in a file named
\cde{top.dep}. Figure~\ref{fig.package} depicts the structure of
package \cde{analysis} from the NASA PVS Library~\cite{nasalib}.

\begin{figure}
  \centering
  \includegraphics[width=13cm]{images/package.png}
  \caption{The structure of a package \cde{analysis}.}
  \label{fig.package}
\end{figure}

\paragraph{Package metadata.} The metadata of a package is defined
in its \cde{top.dep} file located inside folder
\cde{pvsbin}. Table~\ref{tab.bnf} presents the syntax of the metadata
file in BNF-like notation. The topmost symbol is
$\nterm{\cde{metadata}}$, while $\nterm{\cde{theory}}$ and
$\nterm{\cde{package}}$ are terminals representing, respectively,
theory and package names. The metadata comprises two parts, namely, a
header and a body. The header corresponds to a single line with an `/'
symbol followed by a comma-separated list of theory names; these names
correspond to the names of the theories included in the package (in
any order). A body comprises any number of lines, each with either a
package dependency or a theory dependency. A package dependency
describes a dependency from another package and the list of
theories from that external package being used. A theory dependency



\begin{table}
  \centering
  \begin{tabular}{r c p{8cm}}
    $\nterm{\cde{metadata}}$ & ::= & $\nterm{\cde{header}} \; \nterm{\cde{body}}$ \\
    $\nterm{\cde{header}}$ & ::= & `/' $\nterm{\cde{theorylist}}$ \\
    $\nterm{\cde{theorylist}}$ & ::= & $\nterm{\cde{theory}} \mid \nterm{\cde{theory}}$ `,' $\nterm{\cde{theorylist}}$ \\
    $\nterm{\cde{body}}$ & ::= & $(\nterm{\cde{packagedep}} \mid \nterm{\cde{theorydep}})*$ \\
    $\nterm{\cde{packagedep}}$ & ::= & $\nterm{\cde{package}}$ `/' $\nterm{\cde{theorylist}}$ \\
    $\nterm{\cde{theorydep}}$ & ::= & $\nterm{\cde{theory}}$ `:' $\nterm{\cde{qualtheorylist}}$ \\
    $\nterm{\cde{qualtheorylist}}$ & ::= & $\nterm{\cde{qualtheory}} \mid \nterm{\cde{qualtheory}}$ `,' $\nterm{\cde{qualtheorylist}}$ \\
    $\nterm{\cde{qualtheory}}$ & ::= & $(\nterm{\cde{package}}\; \text{`@'})?$ $\nterm{\cde{theory}}$ 
  \end{tabular}
  %\caption{sd}
  \label{tab.bnf}
\end{table}

\begin{table}
  \centering
    \begin{tabular}{ l l p{8cm}}
       \hline
      $<$part\_one$>$ & ::= & ``/'' $<$theories\_list$>$\\
      $<$theories\_list$>$ & ::= &  $<$theory$>$ $|$ $<$theory$>$ ``,'' $<$theories\_list$>$\\
      \\
      $<$part\_two$>$ & ::= &  $<$package\_dep$>$ ``/'' $<$theories\_dep$>$\\
      $<$theories\_dep$>$ & ::= &  $<$theory\_dep$>$ $|$ $<$theory\_dep$>$ ``,'' $<$theories\_dep$>$\\
      \\
      $<$part\_three$>$ & ::= &  $<$theory$>$ ``:'' $<$dependencies\_pkg$>$\\
      $<$dependencies\_pkg$>$ & ::= &  $<$package\_dep$>$ ``@'' $<$theory\_dep$>$ $|$ $<$package\_dep$>$ ``@'' $<$theory\_dep$>$ ``,'' $<$dependencies\_pkg$>$\\
      \hline
    \end{tabular}
    \caption{sd}
  \label{tab.bnf}
\end{table}
