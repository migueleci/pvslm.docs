\section{Installation and Architecture}
\label{sec.install}

This section includes an overview of the installation procedure
and architecture of $\pvslm$.

The $\pvslm$ tool can be installed automatically from the command line
by issuing the following command:
%
\begin{verbatim}
  curl http://migueleci.github.io/pvslm/downloads/pvslm-conf.py \
    -o pvslm-install && chmod +x pvslm-install && \
    python ./pvslm-install
\end{verbatim}
%
This command uses the $\curl$ utility to download the $\pvslm$
installation sources from Github. Once these sources are downloaded
and some file permissions adjusted, the installation process is
executed as a Python 2 script. During the installation process, the
user can select the location in which the tool is to be installed,
including where the configuration files for the library sources and
the local copy of the libraries are to be placed. This script has been
tested both on Linux and Mac OS X boxes. Figure~\ref{fig.install}
depicts a successful installation procedure of $\pvslm$ in a Ubuntu
Linux box.

\begin{figure}
  \centering
  \includegraphics[width=11cm]{images/install.png}
  \caption{A successful installation procedure of $\pvslm$ in a Ubuntu
    Linux box.}
  \label{fig.install}
\end{figure}

Upon installation, $\pvslm$ automatically configures the NASALib
library sources and makes a local copy of them by using Git's clone
command, so they are available for installation in PVS.

The $\pvslm$ is desiged with a distributed architecture. It can
connect to library sources over the internet. Each time a source is
configured, a local copy of the library is downloaded into the host
system as a $\git$ repository. Further updates of the library are
carried out by using $\git$'s pull command. Figure~\ref{fig.arch}
depicts the architecture of $\pvslm$. It is important to note that
although Github is used as $\git$ server, it is also possible to use
other publicly available servers such as BitBucket, for instance.

\begin{figure}
  \centering \includegraphics[width=11cm]{images/arch.png}
  \caption{$\pvslm$ distributed architecture.}
  \label{fig.arch}
\end{figure}
