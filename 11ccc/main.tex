\documentclass[journal,comsoc]{IEEEtran}

\usepackage{amsmath}
\usepackage{amssymb}
\usepackage[justification=justified,font=footnotesize]{caption}
\usepackage{fancyvrb}
\usepackage{graphicx}
\usepackage{hyperref}
\usepackage[utf8x]{inputenc}
\usepackage{color,listings}
\usepackage{relsize}
\usepackage{setspace}
\usepackage{stmaryrd}
%\usepackage{txfonts}
\usepackage{url}
\usepackage[all]{xy}

% predefined sets
\newcommand{\bools}{\mathbb{B}}
\newcommand{\nats}{\mathbb{N}}
\newcommand{\ints}{\mathbb{Z}}
\newcommand{\rats}{\mathbb{Q}}
\newcommand{\reals}{\mathbb{R}}

% fonts for several objects
\newcommand{\fsys}[1]{\mathsf{#1}}
\newcommand{\fset}[1]{\mathtt{#1}}
\newcommand{\fval}[1]{\mathbf{#1}}
\newcommand{\fsub}[1]{{#1}}
\newcommand{\frlnm}[1]{{\sc #1}}
\newcommand{\extfval}[1]{\overline{\fval{#1}}}
\newcommand{\extfsub}[1]{\overline{\fsub{#1}}}
\newcommand{\fsett}[1]{\mathcal{#1}}
\newcommand{\ff}[1]{\mathit{#1}}
\newcommand{\fp}[1]{\mathit{#1}}
\newcommand{\sort}[1]{\textit{#1}}

% abbreviations
\newcommand{\fsF}{\fsys{F}}
\newcommand{\DS}{\fsys{DS}}
\newcommand{\DSL}{{\fsys{DS}(\mathcal{L})}}
\newcommand{\LARR}{\mathcal{L}_A}
\newcommand{\Fterms}[2]{\mathcal{T}(#1,#2)}
\newcommand{\Fforms}[3]{\mathcal{T}(#1,#2,#3)}

% Boolean elements
\newcommand{\eF}{\fset{F}}
\newcommand{\eT}{\fset{T}}

% global sets
\newcommand{\vprop}{\fset{VPROP}}
\newcommand{\prop}{\fset{PROP}}
\newcommand{\vars}{\fsett{X}}
\newcommand{\funcs}{\fsett{F}}
\newcommand{\preds}{\fsett{P}}
\newcommand{\lang}{\fsett{L}}
\newcommand{\arity}{\textit{ar}}
\newcommand{\terms}{\fsett{T}(\vars,\funcs)}
\newcommand{\forms}{\fsett{T}(\vars,\funcs,\preds)}
\newcommand{\srest}[2]{{#2}_{\triangleleft {#1}}}

% logical connectives
\newcommand{\STRUE}{\mathit{true}}
\newcommand{\SFALSE}{\mathit{false}}
\newcommand{\SIFF}{\equiv}
\newcommand{\SXOR}{\not\equiv}
\newcommand{\SOR}{\lor}
\newcommand{\SAND}{\land}
\newcommand{\SNEG}{\neg}
\newcommand{\SIMP}{\rightarrow}
\newcommand{\SCON}{\leftarrow}
\newcommand{\SALL}{\forall}
\newcommand{\SEX}{\exists}

% macros for proposiciones
\newcommand{\TRUE}{\STRUE}
\newcommand{\FALSE}{\SFALSE}
\newcommand{\IFF}[2]{(#1 \SIFF #2)}
\newcommand{\XOR}[2]{(#1 \SXOR #2)}
\newcommand{\OR}[2]{(#1 \SOR #2)}
\newcommand{\AND}[2]{(#1 \SAND #2)}
\newcommand{\NEG}[1]{({\SNEG}#1)}
\newcommand{\IMP}[2]{(#1 \SIMP #2)}
\newcommand{\CON}[2]{(#1 \SCON #2)}
\newcommand{\ALL}[2]{(\SALL{#1}\,{#2})}
\newcommand{\EX}[2]{(\SEX{#1}\,{#2})}

\newcommand{\AIFF}[2]{#1 \SIFF #2}
\newcommand{\AXOR}[2]{#1 \SXOR #2}
\newcommand{\AOR}[2]{#1 \SOR #2}
\newcommand{\AAND}[2]{#1 \SAND #2}
\newcommand{\ANEG}[1]{{\SNEG}#1}
\newcommand{\AIMP}[2]{#1 \SIMP #2}
\newcommand{\ACON}[2]{#1 \SCON #2}
\newcommand{\AALL}[2]{\SALL{#1}\,{#2}}
\newcommand{\AEX}[2]{\SEX{#1}\,{#2}}
\newcommand{\QALL}[3]{\left(\SALL{#1}\mid {#2}: {#3}\right)}
\newcommand{\QEX}[3]{(\SEX{#1}\mid {#2}: {#3})}
\newcommand{\QALLS}[2]{\left(\SALL{#1}\mid : {#2}\right)}
\newcommand{\QEXS}[2]{\left(\SEX{#1}\mid : {#2}\right)}

% calligraphic characters
\newcommand{\CAL}[1]{\mathcal{#1}}
\newcommand{\lC}{\CAL{L}}
\newcommand{\rC}{\CAL{R}}


% other syntax
\newcommand{\DIVS}{{\cdot|}}
\newcommand{\tsub}[3]{#1\!\left[{#2}:={#3}\right]}
\newcommand{\divs}[2]{{#1}\,\DIVS\,{#2}}

% environments
\newenvironment{calc}{\begin{array*}}{\end{array*}}
%\newenvironment{calc}{\begin{align*}}{\end{align*}}
\newcommand{\expr}[1]{ & \; {#1} \\}
\newcommand{\exprnnl}[1]{ & {#1}}
\newcommand{\expl}[2]{#1 & \quad \langle \; \textnormal{#2} \;\rangle \\}

\lstdefinelanguage{Maude}{%
   keywords={
    , mod, fmod, endm, endfm
    , pr , protecting 
    , ex , extending 
    , inc, including
    , sort, sorts, subsort, subsorts
    , var, vars
    , op, ops
    , eq, ceq
    , rl, crl
    , if
    , search
    }
}
\lstnewenvironment{maude}
{\lstset{language=Maude ,
    keywordstyle=\color{blue}
  , basicstyle=\ttfamily\singlespacing\small
  , commentstyle={}
  , columns=flexible
  , numbers=none
  , showstringspaces=false
  , keepspaces=true
  , aboveskip=-3pt
  } 
}
{}

\DefineVerbatimEnvironment{maude2}{Verbatim}{fontsize=\scriptsize}

% rewriting logic definitions
\newcommand{\tsort}[1]{k_{#1}}
\newcommand{\ecal}{\mathcal{E}}
\newcommand{\rcal}{\mathcal{R}}
\newcommand{\tcal}{\mathcal{T}}
%\newcommand{\sortr}[1]{\mathit{#1}}
\newcommand{\func}[3]{#1 : #2 \longrightarrow #3}
\newcommand{\tops}[1]{\left[#1\right]}
\newcommand{\domr}[1]{{\it dom}(#1)}
\newcommand{\ranr}[1]{{\it ran}(#1)}
\newcommand{\varsr}[1]{\textit{vars}(#1)}
\newcommand{\ids}{\textit{id}}
\newcommand{\CSU}{\textit{CSU}}
\newcommand{\GU}{\textit{GU}}
\newcommand{\csu}[3]{\CSU_{#1}(#2 \EQ #3)}
\newcommand{\gsu}[3]{\GU_{#1}(#2 \EQ #3)}
\newcommand{\ls}[1]{\mathit{ls}(#1)}
\newcommand{\sord}[1]{\ll_{#1}}
\newcommand{\posrm}{{\it pos}}
\newcommand{\bool}{\sort{Bool}}
\newcommand{\boolq}{[\bool]}
\newcommand{\true}{\top}
\newcommand{\false}{\bot}
\newcommand{\states}{\mathfrak{s}}
\newcommand{\rews}{\rew}
\newcommand{\rewsn}[1]{\stackrel{#1}{\rew}}
\newcommand{\Pip}{\Pi_{s_1,\ldots,s_n}}
\newcommand{\IND}{\mathit{ind}}
\newcommand{\can}[2]{{#1}\downarrow_{#2}}
\newcommand{\Can}{{\bf Can}}
\newcommand{\rew}{\rightarrow}
\newcommand{\ded}{\vdash}
\newcommand{\ided}{\ded_\IND}
\newcommand{\vdashind}{\ided}
\newcommand{\gded}{\Vdash}
\newcommand{\Crm}{{\it C}}
\newcommand{\Normrm}{{\it Norm}}
\newcommand{\Canrm}{{\it Can}}
\newcommand{\reachrm}{{\it reach}}
\newcommand{\cans}[1]{\overline{#1}}
\newcommand{\steps}[2]{#1^{#2}}
\newcommand{\clos}[2]{\stackrel{#1}{#2}}
\newcommand{\rasy}[1]{\clos{\vartriangle}{#1}}
\newcommand{\rpar}[1]{\clos{\shortparallel}{#1}}
\newcommand{\subst}[3]{#1 : #2 \longrightarrow #3}

% auxiliary notation and functions for equations and rules
\newcommand{\EQ}{=}
\newcommand{\IFS}{\mathbf{if}}
\newcommand{\REW}{\rightarrow}
\newcommand{\eq}[2]{#1 \EQ #2}
\newcommand{\ceq}[3]{#1 \EQ #2 \; \IFS\; #3}
\newcommand{\rl}[2]{#1 \REW #2}
\newcommand{\crl}[3]{#1 \REW #2 \; \IFS\; #3}
\newcommand{\lbl}[1]{[\mathit{#1}]}
\newcommand{\cond}{\gamma}

\newcommand{\pvslm}{\textsf{pvslm}}


\usepackage[T1]{fontenc}% optional T1 font encoding
\usepackage{amsmath}
\interdisplaylinepenalty=2500
\usepackage[cmintegrals]{newtxmath}

% correct bad hyphenation here
\hyphenation{op-tical net-works semi-conduc-tor}

\begin{document}

\title{Library Management for PVS}

\author{
\IEEEauthorblockN{Miguel Romero\IEEEauthorrefmark{1}, Camilo Rocha\IEEEauthorrefmark{2}\\}
    \IEEEauthorblockA{\IEEEauthorrefmark{1}Escuela Colombiana de Ingenier\'{i}a - Bogot\'{a}, Colombia. Email: miguel.romero@mail.escuelaing.edu.co\\}
    \IEEEauthorblockA{\IEEEauthorrefmark{2}Pontificia Universidad Javeriana - Cali, Colombia. Email: camilo.rocha@javerianacali.edu.co}
}

\maketitle

\begin{abstract}
The Prototype Verification System (PVS) is a successful specification
language with support tools and an automated theorem prover that is
being used globally by the formal methods community. This paper
presents $\pvslm$, a tool for managing PVS libraries featuring support
for different library sources, libraries with several theories, and
dependencies among libraries and theories (even from different library
sources). The tool, freely available for download, is a command line
application written in the Python programming language and depends,
mainly, on the distributed revision control system Git. This paper
presents the main features of $\pvslm$ and a description language for
annotating libraries so that they can be shared with the tool.  This
paper also includes a detailed step-by-step guide on how to use
$\pvslm$ for the configuration of the current version of the NASA PVS
Library, available at GitHub.
\end{abstract}


\begin{IEEEkeywords}
PVS, PVSLM, PVS Library Manager, PVS theory, Git, Python
\end{IEEEkeywords}

% For peerreview papers, this IEEEtran command inserts a page break and
% creates the second title. It will be ignored for other modes.
\IEEEpeerreviewmaketitle

\section{Introduction}
\label{sec.intro}


The Prototype Verification System~\cite{pvs-cade92} (PVS) is a
verification system comprising a specification language integrated
with support tools and a theorem prover; basically, PVS is a
mechanization of classical typed higher-order logic with
specifications organized into parameterized theories in this logic.
The PVS system has been used in state-of-the-art formal methods
projects as a productive environment for constructing and maintaining
collections of theories, both in industry and in research. These novel
developments often result in large collections of PVS definitions,
theorems, and proofs, grouped into libraries, with many dependencies
among them. This is the case, for instance, of the NASA PVS
Library~\cite{nasalib} maintained by the NASA Langley Formal Methods
Team: a large collection of freely-available formal developments in
PVS comprising more than 40 theories and xx theorems, ranging from
trigonometry to graphs and topology.


\section{Library Configuration}
\label{sec.conf}

This section describes the conventions and assumptions made by
$\pvslm$ for managing PVS library sources.

\paragraph{Conventions.}
The $\pvslm$ tool distinguishes three levels of aggregation for PVS
sources. A PVS theory is the building block of a library managed by
$\pvslm$. A {\em package} is a collection of theories. At the top
level of aggregation is a {\em library}, which comprises a collection
of packages. Namely, a library is a collection of packages and a
package is a collection of libraries. The $\pvslm$ tool can manage
several libraries each with several packages. However, the tool does
not support management at the level of specific theory files.

\paragraph{Package configuration.}
Each package of a library is defined in a folder at the root of the
library source. The folder's name corresponds to the name of the
package. Each folder contains the corresponding \cde{pvs} and
\cde{prf} files for its theories, and the folder \cde{pvsbin}, a
special folder with the package's metadata described in a file named
\cde{top.dep}. Figure~\ref{fig.package} depicts the structure of
package \cde{analysis} from the NASA PVS Library~\cite{nasalib}
(NASALib).

\begin{figure}
  \centering
  \includegraphics[width=11cm]{images/package.png}
  \caption{The structure of a package \cde{analysis}.}
  \label{fig.package}
\end{figure}

\paragraph{Package metadata.} The metadata of a package is defined
in its \cde{top.dep} file located inside folder
\cde{pvsbin}. Table~\ref{tab.bnf} presents the syntax of the metadata
file in BNF-like notation. The topmost symbol is
$\nterm{\cde{metadata}}$, while $\nterm{\cde{theory}}$ and
$\nterm{\cde{package}}$ are terminals representing, respectively,
theory and package names. The metadata comprises two parts, namely, a
header and a body. The header corresponds to a single line with an `/'
symbol followed by a comma-separated list of theory names; these names
correspond to the names of the theories included in the package (in
any order). A body comprises any number of lines, each with either a
package dependency or a theory dependency. A package dependency
describes a dependency from another package and the list of theories
from that external package being used. A theory dependency describes,
for each one of the theories listed in the header of the package, its
list of theory dependencies. In the case such a theory depends on a
theory from other package, the name of that dependency must be
qualified by the name of the corresponding package.

\begin{table}
  \centering
  \begin{tabular}{r c p{8cm}}
    \hline \\
    $\nterm{\cde{metadata}}$ & ::= & $\nterm{\cde{header}} \; \nterm{\cde{body}}$ \\
    $\nterm{\cde{header}}$ & ::= & `/' $\nterm{\cde{theorylist}}$ \\
    $\nterm{\cde{theorylist}}$ & ::= & $\nterm{\cde{theory}} \mid \nterm{\cde{theory}}$ `,' $\nterm{\cde{theorylist}}$ \\
    $\nterm{\cde{body}}$ & ::= & $(\nterm{\cde{packagedep}} \mid \nterm{\cde{theorydep}})*$ \\
    $\nterm{\cde{packagedep}}$ & ::= & $\nterm{\cde{package}}$ `/' $\nterm{\cde{theorylist}}$ \\
    $\nterm{\cde{theorydep}}$ & ::= & $\nterm{\cde{theory}}$ `:' $\nterm{\cde{qualtheorylist}} ?$ \\
    $\nterm{\cde{qualtheorylist}}$ & ::= & $\nterm{\cde{qualtheory}} \mid \nterm{\cde{qualtheory}}$ `,' $\nterm{\cde{qualtheorylist}}$ \\
    $\nterm{\cde{qualtheory}}$ & ::= & $(\nterm{\cde{package}}\; \text{`@'})?$ $\nterm{\cde{theory}}$ \\
    \\
    \hline
  \end{tabular}
  \caption{Syntax of the \cde{top.dep} metadata file.}
  \label{tab.bnf}
\end{table}

Figure~\ref{fig.top} presents an overview of the configuration file
for package \cde{trig} in NASALib. According to the header
description, package \cde{trig} defines theories \cde{top},
\cde{trig\_doc}, \cde{trig}, \cde{trig\_values}, etc. It contains $5$
package dependencies and $7$ theory dependencies. For instance,
package \cde{trig} depends on theory \cde{for\_iterate} from package
\cde{structures}, and on theories \cde{finite\_sets\_minmax} and
\cde{finite\_sets\_inductions} from package \cde{finite\_sets}. On the
other hand, theory \cde{trig\_doc} has no dependencies, while theory
\cde{trig} depends on packages \cde{trig\_basic}, \cde{sqrt},
\cde{trig\_values}, and \cde{trig\_ineq}. In the case of theory
\cde{sqrt}, it is explicitly stated that such a theory comes
from package \cde{reals}.

\begin{figure}
  \centering
  \includegraphics[width=11cm]{images/top.png}
  \caption{Overview of metadata file for package \cde{trig} in NASALib.}
  \label{fig.top}
\end{figure}


\section{Obtaining the $\pvslm$ Implementation}
\label{sec.install}

The $\pvslm$ tool can be installed automatically from the terminal in
*nix systems by issuing the following command:
%
{\small\begin{verbatim}
  curl http://migueleci.github.io/pvslm/downloads/pvslm-conf.py \
    -o pvslm-install && chmod +x pvslm-install && \
    python ./pvslm-install
\end{verbatim}}
%
This command uses the $\curl$ utility to download the $\pvslm$
installation script. Once downloaded, the installation script is
executed as a Python 2 program.  During the installation process, the
user can select the value for global variable \cde{PVS\_PATH}, i.e.,
the location in which the tool is to be installed, including where the
configuration files for the library sources and the local copy of the
libraries are to be placed. This procedure and the installation script
have been tested both on Linux and Mac OS X boxes. It is important to
note that the installation script, as it is also the case for the
$\pvslm$ implementation, relies on the $\git$ utility --a free, open,
and popular source distributed version control system--.

Upon its successful installation, the $\pvslm$ installation script
automatically configures the NASALib library sources and makes a local
copy of the entire library, so that they are immediately available for
installation in PVS.  Further information on the installation
procedure is available from~\cite{pvslm}, including a detailed list of
the tool commands and some examples of configuration files.


\section{Available Commands}
\label{sec.cmd}

This section presents the list commands available from the $\pvslm$
tool. Section~\ref{sec.nasalib} presents examples on the use of some
of these commands.

The $\pvslm$ tool provides commands at two levels. First, it provides
commands at the level of library sources for managing library $\git$
repositories.  Second, it provides commands at the level of packages
for managing the contents of a library. Table~\ref{tab.cmd} lists all
commands available from $\pvslm$.

At the level of library sources, the tool provides $5$ different
commands, all identified with the special token \cde{src}. They
include commands for:

\begin{itemize}
  \item Adding a library source (i.e., command \cde{-a}) with a
    name, a short description, and a $\git$ URL.
  \item Deleting a library source (i.e., command \cde{-d}) with
    the given name.
  \item Cloning a library source (i.e., command \cde{-c}) with the
    given name.
  \item Updating a library source (i.e., command \cde{-u}) with
    the given name.
  \item Removing the clone of a library source (i.e., command
    \cde{-r}) with the given name.
\end{itemize}

It is important to note that none of the commands at the library
source level modify the user's PVS installation. These commands
exclusively modify the internals of the $\pvslm$ configuration. Also,
note that a library source is realized exactly by one $\git$
repository.

At the level of packages, the tool provides $4$ different
commands, all identified with the special token \cde{pkg}. They
include commands for:

\begin{itemize}
  \item Installing and updating a given package (i.e., command
    \cde{-i}) from a given library source, including {\em all} its
    dependencies.
  \item Updating a given package (i.e., command \cde{-u}) from a given
    library source, including {\em all} its dependencies.
  \item Deleting a given package (i.e., command \cde{-d}) from a given
    library source (local copy), including all packages that depend on
    it.
  \item Listing the contents (i.e., command \cde{-l}) from a given
    library source.
\end{itemize}

Note that listing command has three variants. In the first case, all
libraries available to the system are listed. In the second case, all
packages of a given library are listed. In the third case, all
dependencies of a given package and library are listed. Internally,
the $\pvslm$ uses a topological sorting algorithm, based upon each
package's metadata, for computing the set of dependencies among
theories and packages.

\begin{table*}[pthb]
  \begin{center}
    \begin{tabular}{ | l | l | l | p{6cm} | }
      \hline Level & Command & Parameters & Description \\ \hline
      src & -a          & name desc URL         & Add a new library source with the given name, description, and URL.                   \\ \cline{2-4}
          & -d          & name                  & Delete the given library source.                                          \\ \cline{2-4}
          & -c          & name                  & Clone the given library source.                          \\ \cline{2-4}
          & -u          & name                  & Update the given library source.                                \\ \cline{2-4}
          & -r          & name                  & Remove the clone of the given library source.                                                  \\ \hline
      pkg & -i          & library@package       & Install and update the given package, and all its dependencies.                \\ \cline{2-4}
          & -u          & library@package       & Update a package and all its dependencies.                   \\ \cline{2-4}
          & -d          & library@package       & Delete a  package and all ones depending on it.           \\ \cline{2-4}
          & -l          &                       & List the installed libraries.                                       \\ \cline{2-4}
          & -l          & library               & List the available packages of the given library.                         \\ \cline{2-4}
          & -l          & library@package       & List all the dependencies of the given package.                               \\ \hline
    \end{tabular}
  \end{center}
  \caption{$\pvslm$ command list.}
  \label{tab.cmd}
\end{table*}



\section{Nasalib Configuration}
\label{sec.nasalib}


\section{Concluding Remarks}
\label{sec.concl}

This paper presented the $\pvslm$ tool for managing libaries for the
Prototype Verification System (PVS). This tool features support for
different library sources, libraries with several theories, and
dependencies among the theories (within the same library source).  The
tool is freely available for download and it is distributed under
GNU's GPLv3 license. The tool uses a small footprint language for
annotating libraries, which is described in full detail in BNF-like
notation in this paper. This paper also includes all commands
available from the tool, its architecture, and an overview of its
installation process.  A detailed step-by-step case study is included
for illustrating the main features of the tool.

As usual, much work remains to be done. First, it is important to make
available other library sources through the $\pvslm$ tool.  Also, it
is important to test the tool against different servers and in more
operating systems. Finally, it would be highly desirable for the tool
to manage different versions of a library source, which can depend on
the installed versions of PVS in the host system. This will likely
require an extension of the current metadata language for annotating
library sources.

\paragraph{\bf Acknowledgments.} The authors would like to thank
C. Mu\~noz from the NASA Langley Formal Methods Team for his
encouragement, ideas, and suggestions, specially for the help with the
definition of the metadata description language for packages.


\section*{Acknowledgment}
The authors would like to thank
C. Mu\~noz in the NASA Langley Formal Methods Team for his
encouragement, ideas, and suggestions, specially for the help with the
definition of the metadata description language for packages.

% Can use something like this to put references on a page
% by themselves when using endfloat and the captionsoff option.
\ifCLASSOPTIONcaptionsoff
  \newpage
\fi

\bibliographystyle{IEEEtran}
\bibliography{biblio}

\end{document}


