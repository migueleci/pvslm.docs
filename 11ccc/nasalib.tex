\section{Case study: Managing NASALib with $\pvslm$}
\label{sec.nasalib}

This section presents a step-by-step guide on the configuration and
installation with $\pvslm$ of the NASA PVS Library (NASALib) that is
available from GitHub.

\paragraph{Library source configuration}
The first step is to issue a command for configuring the library sources
as follows:
%
{\small
\begin{lstlisting}
$ pvslm.py src -a \\
   nasalib \\
   `The NASA PVS Library is a collection of formal PVS developments \\
    maintained by the NASA Langley Formal Methods Team.' \\
   https://github.com/nasa/pvslib.git
\end{lstlisting}
}
%
Internally, this command generates the library source
configuration file in the destination folder chosen during the
installation process of $\pvslm$.

It is important to note that this step is unnecessary with the current
distribution of $\pvslm$ because it installs the NASALib library
source by default.  However, this command is included in this section
for the sake of a complete installation example. Also note that it is
not possible to have two or more library sources with the same name.

\paragraph{Cloning a library} After a library source
is configured, it is possible to issue a command for cloning the
library source packages into a local $\git$ repository. The following
command clones NASALib:
%
\begin{verbatim}
$ pvslm.py src -c nasalib
\end{verbatim}
%
This command internally uses $\git$'s clone command to obtain a local
copy of the entire library source repository by using the URL
configured with the library source.

\paragraph{Updating a library} When necessary, it is possible to update
the contents of a local copy of a library source. For the effect of
updating the local copy of NASALib, a user can issue the following
command:
%
\begin{verbatim}
$ pvslm.py src -u nasalib
\end{verbatim}
%
The effect of this command is to update the local copy of the entire
library via $\git$'s pull command by using the URL associated to
the library source.

\paragraph{Listing the packages in a library} By having a local
copy of a library, it is possible to list all its packages. Also, it
is possible to list the dependencies of a given package. The following
command lists the package dependencies of package \cde{complex} from
NASALib:
%
\begin{verbatim}
$ pvslm.py pkg -l nasalib@complex
\end{verbatim}
%
This command generates the following output:
%
\begin{verbatim}
Package complex depends on:
algebra
analysis_ax
ints
lnexp
reals
structures
trig
\end{verbatim}

\paragraph{Installing a package} The following command installs package
\cde{complex} in PVS.
%
\begin{verbatim}
$ pvslm.py pkg -i nasalib@complex
\end{verbatim}
%
Since package \cde{complex} depends on other packages, the $\pvslm$
tool also asks for permission to install its dependencies. The
following is the output to the user for this installation command.
%
\begin{lstlisting}
Package complex depends on:
...
Would you like to install the package(s) (y/N): 
\end{lstlisting}
%
By convention, $\pvslm$ performs {\em all} installations in the folder
specified by the \cde{PVS\_PATH} global variable.

\paragraph{Deleting a package} Finally, a user can issue the following
command to delete package \cde{ints} from the (internal) local copy of
NASALib:
%
\begin{verbatim}
$ pvslm.py pkg -d nasalib@ints
\end{verbatim}
%
If there are packages that depend on the package to be removed, the $\pvslm$ tool
lists all of them and asks the user for authorization to remove them
too.
