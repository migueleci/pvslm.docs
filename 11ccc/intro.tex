\section{Introduction}
\label{sec.intro}


The Prototype Verification System~\cite{pvs-cade92} (PVS) is a
verification system comprising a specification language integrated
with support tools and a theorem prover; basically, PVS is a
mechanization of classical typed higher-order logic with
specifications organized into parameterized theories in this logic.
The PVS system has been used in state-of-the-art formal methods
projects as a productive environment for constructing and maintaining
collections of theories, both in industry and in research. These novel
developments often result in large collections of PVS definitions,
theorems, and proofs, grouped into libraries, with many dependencies
among them. This is the case, for instance, of the NASA PVS
Library~\cite{nasalib} maintained by the NASA Langley Formal Methods
Team: a large collection of freely-available formal developments in
PVS comprising more than 40 theories, 24000+ theorems, and 136000+ and
3623000+ lines of, respectively, specification and proofs, all of
theses ranging from trigonometry to graphs and topology.  In general,
given the increasing size of mathematical developments in the form of
libraries and the growing need for and adoption of mechanized proof
environments~\cite{avigad-mech14,hales-proofs14} such as PVS, it is
important to have tools for managing such libraries.

This paper presents $\pvslm$~\cite{pvslm}, an utility for assisting in
the management of PVS libraries, comprising: (i) a description
language for annotating PVS libraries, with a small footprint but
expressive enough for describing complex dependencies among library
theories; and (ii) an implementation for managing PVS libraries
annotated with this language, offering support for several library
sources. In $\pvslm$, the description language is designed to
represent a PVS library as a collection of packages, with a package
consisting of a collection of PVS theories. In this sense, the
annotations can help in documenting each library with the information
of the theories it offers and their dependencies, so it can be shared
among PVS users with the help of the $\pvslm$ implementation. The
$\pvslm$ implementation is a command line tool offering support for
adding, updating, deleting, and re-installing the contents of a
library source, and also managing several library sources at the same
time.

The $\pvslm$ description language is presented in this paper with the
help of BNF-like notation. The $\pvslm$ implementation is written in
the Python programming language and it can manage {\em any}
(annotated) PVS library that is publicly available from a $\pvslm$
server through the internet: once available, such a library source can
be configured in $\pvslm$, be automatically downloaded from the
internet, and set up in the host system. As a case study for the use
of the $\pvslm$ tool, this paper presents a step-by-step guide on how
to manually configure the NASA PVS Library comprising the command line
instructions and snapshots of the user interaction.

The current distribution of \verb$pvslm$ is freely available for
download~\cite{pvslm} under the GNU General Public License GPLv3 and
it automatically installs the latest version of the NASA PVS Library,
which is currently annotated with the $\pvslm$ description language.

\paragraph{Paper outline.} Section ~\ref{sec.conf} presents the $\pvslm$ 
description language. Sections ~\ref{sec.install} and ~\ref{sec.cmd} 
present, respectively, instructions for obtaining and installing the $\pvslm$ 
implementation, and a list of commands available from $\pvslm$. 
Section ~\ref{sec.nasalib} includes a step-by-step guide on the configuration and
installation with $\pvslm$ of the NASA PVS Library. Some concluding
remarks are presented in Section ~\ref{sec.concl}.
