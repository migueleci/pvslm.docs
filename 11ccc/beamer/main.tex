\documentclass[mathserif,fleqn]{beamer}


\usepackage{amsmath}
\usepackage{amssymb}
\usepackage[justification=justified,font=footnotesize]{caption}
\usepackage{fancyvrb}
\usepackage{graphicx}
\usepackage{hyperref}
\usepackage[utf8x]{inputenc}
\usepackage{color,listings}
\usepackage{relsize}
\usepackage{setspace}
\usepackage{stmaryrd}
%\usepackage{txfonts}
\usepackage{url}
\usepackage[all]{xy}

% predefined sets
\newcommand{\bools}{\mathbb{B}}
\newcommand{\nats}{\mathbb{N}}
\newcommand{\ints}{\mathbb{Z}}
\newcommand{\rats}{\mathbb{Q}}
\newcommand{\reals}{\mathbb{R}}

% fonts for several objects
\newcommand{\fsys}[1]{\mathsf{#1}}
\newcommand{\fset}[1]{\mathtt{#1}}
\newcommand{\fval}[1]{\mathbf{#1}}
\newcommand{\fsub}[1]{{#1}}
\newcommand{\frlnm}[1]{{\sc #1}}
\newcommand{\extfval}[1]{\overline{\fval{#1}}}
\newcommand{\extfsub}[1]{\overline{\fsub{#1}}}
\newcommand{\fsett}[1]{\mathcal{#1}}
\newcommand{\ff}[1]{\mathit{#1}}
\newcommand{\fp}[1]{\mathit{#1}}
\newcommand{\sort}[1]{\textit{#1}}

% abbreviations
\newcommand{\fsF}{\fsys{F}}
\newcommand{\DS}{\fsys{DS}}
\newcommand{\DSL}{{\fsys{DS}(\mathcal{L})}}
\newcommand{\LARR}{\mathcal{L}_A}
\newcommand{\Fterms}[2]{\mathcal{T}(#1,#2)}
\newcommand{\Fforms}[3]{\mathcal{T}(#1,#2,#3)}

% Boolean elements
\newcommand{\eF}{\fset{F}}
\newcommand{\eT}{\fset{T}}

% global sets
\newcommand{\vprop}{\fset{VPROP}}
\newcommand{\prop}{\fset{PROP}}
\newcommand{\vars}{\fsett{X}}
\newcommand{\funcs}{\fsett{F}}
\newcommand{\preds}{\fsett{P}}
\newcommand{\lang}{\fsett{L}}
\newcommand{\arity}{\textit{ar}}
\newcommand{\terms}{\fsett{T}(\vars,\funcs)}
\newcommand{\forms}{\fsett{T}(\vars,\funcs,\preds)}
\newcommand{\srest}[2]{{#2}_{\triangleleft {#1}}}

% logical connectives
\newcommand{\STRUE}{\mathit{true}}
\newcommand{\SFALSE}{\mathit{false}}
\newcommand{\SIFF}{\equiv}
\newcommand{\SXOR}{\not\equiv}
\newcommand{\SOR}{\lor}
\newcommand{\SAND}{\land}
\newcommand{\SNEG}{\neg}
\newcommand{\SIMP}{\rightarrow}
\newcommand{\SCON}{\leftarrow}
\newcommand{\SALL}{\forall}
\newcommand{\SEX}{\exists}

% macros for proposiciones
\newcommand{\TRUE}{\STRUE}
\newcommand{\FALSE}{\SFALSE}
\newcommand{\IFF}[2]{(#1 \SIFF #2)}
\newcommand{\XOR}[2]{(#1 \SXOR #2)}
\newcommand{\OR}[2]{(#1 \SOR #2)}
\newcommand{\AND}[2]{(#1 \SAND #2)}
\newcommand{\NEG}[1]{({\SNEG}#1)}
\newcommand{\IMP}[2]{(#1 \SIMP #2)}
\newcommand{\CON}[2]{(#1 \SCON #2)}
\newcommand{\ALL}[2]{(\SALL{#1}\,{#2})}
\newcommand{\EX}[2]{(\SEX{#1}\,{#2})}

\newcommand{\AIFF}[2]{#1 \SIFF #2}
\newcommand{\AXOR}[2]{#1 \SXOR #2}
\newcommand{\AOR}[2]{#1 \SOR #2}
\newcommand{\AAND}[2]{#1 \SAND #2}
\newcommand{\ANEG}[1]{{\SNEG}#1}
\newcommand{\AIMP}[2]{#1 \SIMP #2}
\newcommand{\ACON}[2]{#1 \SCON #2}
\newcommand{\AALL}[2]{\SALL{#1}\,{#2}}
\newcommand{\AEX}[2]{\SEX{#1}\,{#2}}
\newcommand{\QALL}[3]{\left(\SALL{#1}\mid {#2}: {#3}\right)}
\newcommand{\QEX}[3]{(\SEX{#1}\mid {#2}: {#3})}
\newcommand{\QALLS}[2]{\left(\SALL{#1}\mid : {#2}\right)}
\newcommand{\QEXS}[2]{\left(\SEX{#1}\mid : {#2}\right)}

% calligraphic characters
\newcommand{\CAL}[1]{\mathcal{#1}}
\newcommand{\lC}{\CAL{L}}
\newcommand{\rC}{\CAL{R}}


% other syntax
\newcommand{\DIVS}{{\cdot|}}
\newcommand{\tsub}[3]{#1\!\left[{#2}:={#3}\right]}
\newcommand{\divs}[2]{{#1}\,\DIVS\,{#2}}

% environments
\newenvironment{calc}{\begin{array*}}{\end{array*}}
%\newenvironment{calc}{\begin{align*}}{\end{align*}}
\newcommand{\expr}[1]{ & \; {#1} \\}
\newcommand{\exprnnl}[1]{ & {#1}}
\newcommand{\expl}[2]{#1 & \quad \langle \; \textnormal{#2} \;\rangle \\}

\lstdefinelanguage{Maude}{%
   keywords={
    , mod, fmod, endm, endfm
    , pr , protecting 
    , ex , extending 
    , inc, including
    , sort, sorts, subsort, subsorts
    , var, vars
    , op, ops
    , eq, ceq
    , rl, crl
    , if
    , search
    }
}
\lstnewenvironment{maude}
{\lstset{language=Maude ,
    keywordstyle=\color{blue}
  , basicstyle=\ttfamily\singlespacing\small
  , commentstyle={}
  , columns=flexible
  , numbers=none
  , showstringspaces=false
  , keepspaces=true
  , aboveskip=-3pt
  } 
}
{}

\DefineVerbatimEnvironment{maude2}{Verbatim}{fontsize=\scriptsize}

% rewriting logic definitions
\newcommand{\tsort}[1]{k_{#1}}
\newcommand{\ecal}{\mathcal{E}}
\newcommand{\rcal}{\mathcal{R}}
\newcommand{\tcal}{\mathcal{T}}
%\newcommand{\sortr}[1]{\mathit{#1}}
\newcommand{\func}[3]{#1 : #2 \longrightarrow #3}
\newcommand{\tops}[1]{\left[#1\right]}
\newcommand{\domr}[1]{{\it dom}(#1)}
\newcommand{\ranr}[1]{{\it ran}(#1)}
\newcommand{\varsr}[1]{\textit{vars}(#1)}
\newcommand{\ids}{\textit{id}}
\newcommand{\CSU}{\textit{CSU}}
\newcommand{\GU}{\textit{GU}}
\newcommand{\csu}[3]{\CSU_{#1}(#2 \EQ #3)}
\newcommand{\gsu}[3]{\GU_{#1}(#2 \EQ #3)}
\newcommand{\ls}[1]{\mathit{ls}(#1)}
\newcommand{\sord}[1]{\ll_{#1}}
\newcommand{\posrm}{{\it pos}}
\newcommand{\bool}{\sort{Bool}}
\newcommand{\boolq}{[\bool]}
\newcommand{\true}{\top}
\newcommand{\false}{\bot}
\newcommand{\states}{\mathfrak{s}}
\newcommand{\rews}{\rew}
\newcommand{\rewsn}[1]{\stackrel{#1}{\rew}}
\newcommand{\Pip}{\Pi_{s_1,\ldots,s_n}}
\newcommand{\IND}{\mathit{ind}}
\newcommand{\can}[2]{{#1}\downarrow_{#2}}
\newcommand{\Can}{{\bf Can}}
\newcommand{\rew}{\rightarrow}
\newcommand{\ded}{\vdash}
\newcommand{\ided}{\ded_\IND}
\newcommand{\vdashind}{\ided}
\newcommand{\gded}{\Vdash}
\newcommand{\Crm}{{\it C}}
\newcommand{\Normrm}{{\it Norm}}
\newcommand{\Canrm}{{\it Can}}
\newcommand{\reachrm}{{\it reach}}
\newcommand{\cans}[1]{\overline{#1}}
\newcommand{\steps}[2]{#1^{#2}}
\newcommand{\clos}[2]{\stackrel{#1}{#2}}
\newcommand{\rasy}[1]{\clos{\vartriangle}{#1}}
\newcommand{\rpar}[1]{\clos{\shortparallel}{#1}}
\newcommand{\subst}[3]{#1 : #2 \longrightarrow #3}

% auxiliary notation and functions for equations and rules
\newcommand{\EQ}{=}
\newcommand{\IFS}{\mathbf{if}}
\newcommand{\REW}{\rightarrow}
\newcommand{\eq}[2]{#1 \EQ #2}
\newcommand{\ceq}[3]{#1 \EQ #2 \; \IFS\; #3}
\newcommand{\rl}[2]{#1 \REW #2}
\newcommand{\crl}[3]{#1 \REW #2 \; \IFS\; #3}
\newcommand{\lbl}[1]{[\mathit{#1}]}
\newcommand{\cond}{\gamma}

\newcommand{\pvslm}{\textsf{pvslm}}


\hypersetup{pdfstartview={FitH}}
\setbeamertemplate{frametitle}[default][center]
\useinnertheme{rectangles}
%\useoutertheme{infolines}

\definecolor{niceblue}{RGB}{214,214,240}
\definecolor{nicebluefg}{RGB}{51,51,179}

% title page
\setbeamercolor{title}{bg=niceblue}
\setbeamerfont{title}{series=\bfseries}
% frame title
\setbeamercolor{frametitle}{bg=niceblue}
\setbeamerfont{frametitle}{series=\bfseries}
% block title
\setbeamerfont{block title}{series=\bfseries}
\setbeamerfont{section in toc}{series=\bfseries}
% footer
\setbeamertemplate{navigation symbols}{}
% alerted text
\setbeamercolor{alerted text}{fg=nicebluefg}
\setbeamerfont{alerted text}{series=\bfseries}

% custom itemization and enumeration environments
\newenvironment{customitemize}[1][0]
  { \begin{itemize}
    % set spacing between items
    \addtolength{\itemsep}{#1\baselineskip}
    % set spacing between lines
    %\addtolength{\baselineskip}{#1\baselineskip} 
  }
  { \end{itemize} }
\newenvironment{customenumerate}[1][0]
  { \begin{enumerate}
    % set spacing between items
    \addtolength{\itemsep}{#1\baselineskip}
    % set spacing between lines
    %\addtolength{\baselineskip}{#1\baselineskip} 
  }
  { \end{enumerate} }

\newenvironment{outeritemize}
  { \begin{customitemize}[0.5] }
  { \end{customitemize} }
\newenvironment{inneritemize}
  { \begin{customitemize}[0.2] }
  { \end{customitemize} }
  
\newenvironment{outerenumerate}
  { \begin{customenumerate}[0.5] }
  { \end{customenumerate} }
\newenvironment{innerenumerate}
  { \begin{customenumerate}[0.2] }
  { \end{customenumerate} }

\newcommand\Mark[1]{\textsuperscript#1}

\title{ 
  Library Management for PVS
}

\author[M. Romero]{
  {\bf Miguel Romero\Mark{1}} \and Camilo Rocha\Mark{2}
}

\institute[ECI]{
  \large \Mark{1}Escuela Colombiana de Ingenier\'ia Julio Garavito, Bogotá\\ \Mark{2}Pontificia Universidad Javeriana, Cali 
}

\date[MPL 09.27.2016]{
  11CCC\\
  September 29, 2016
}

% PDF settings
\hypersetup{%
  pdftitle=\@title,%
  pdfauthor=\@author%
}	

\begin{document}

\begin{frame}
  \titlepage
\end{frame}

% pending Context
\begin{frame}
  \frametitle{Context}
 
   \begin{outeritemize}
   \item The Prototype Verification System (PVS) is a verification
     system comprising a specification language integrated with
     support tools and a theorem prover.
     
   \item The PVS system has been used in state-of-the-art formal
     methods projects as a productive environment for constructing and
     maintaining collections of theories, both in industry and in
     research.
     
   \item Given the increasing size of mathematical developments in the
     form of libraries and the growing need for and adoption of
     mechanized proof environments such as PVS, it is important to
     have tools for managing such libraries.
   \end{outeritemize}
\end{frame}

% pending Problem
\begin{frame}
  \frametitle{The Problem} There does not exist an utility for the
  management of PVS libraries.
\end{frame}

% pending Contribution
\begin{frame}
  \frametitle{Main Contribution}
   \begin{outeritemize}
   \item \cde{pvslm} is an utility for assisting in the management of
     PVS libraries, comprising:
   \begin{inneritemize}
      \item a description language for annotating PVS libraries
      \item an implementation for managing PVS libraries annotated with this language
   \end{inneritemize}
 \item The description language is designed to represent a PVS library
   as a collection of packages, with a package consisting of a
   collection of PVS theories.

 \item The implementation is written in the Python programming
   language and it can manage any (annotated) PVS library that is
   publicly available from a \cde{pvslm} server through the internet.
   \end{outeritemize}
\end{frame}

% Agenda
\begin{frame}
  \frametitle{Agenda}
  \tableofcontents
\end{frame}

% Description Language
\section{The Description Language}

\begin{frame}
  \frametitle{Agenda} 
  \tableofcontents[currentsection]
\end{frame}

\begin{frame}
  \frametitle{Description Language}

The \cde{pvslm} tool distinguishes three levels of aggregation for PVS
sources:

\begin{itemize}
  \item A {\em theory} is the building block of a library.

  \item A {\em package} is a collection of theories.

  \item A {\em library} is at the top level of aggregation, comprising
    a collection of packages.
\end{itemize}

\end{frame}

\begin{frame}
  \frametitle{Description Language: Syntax}
  {\small
    \begin{table}
    \begin{tabular}{r c p{6cm}}
      \hline \\
      $\nterm{\cde{metadata}}$ & ::= & $\nterm{\cde{header}} \; \nterm{\cde{body}}$ \\
      $\nterm{\cde{header}}$ & ::= & `/' $\nterm{\cde{theorylist}}$ \\
      $\nterm{\cde{theorylist}}$ & ::= & $\nterm{\cde{theory}} \mid \nterm{\cde{theory}}$ `,' $\nterm{\cde{theorylist}}$ \\
      $\nterm{\cde{body}}$ & ::= & $(\nterm{\cde{packagedep}} \mid \nterm{\cde{theorydep}})*$ \\
      $\nterm{\cde{packagedep}}$ & ::= & $\nterm{\cde{package}}$ `/' $\nterm{\cde{theorylist}}$ \\
      $\nterm{\cde{theorydep}}$ & ::= & $\nterm{\cde{theory}}$ `:' $\nterm{\cde{qualtheorylist}} ?$ \\
      $\nterm{\cde{qualtheorylist}}$ & ::= & $\nterm{\cde{qualtheory}} \mid \nterm{\cde{qualtheory}}$ `,' $\nterm{\cde{qualtheorylist}}$ \\
      $\nterm{\cde{qualtheory}}$ & ::= & $(\nterm{\cde{package}}\; \text{`@'})?$ $\nterm{\cde{theory}}$ \\
      \\
      \hline
    \end{tabular}
    \caption{Syntax of the description language in BNF-like notation.}
  \end{table}
  }
\end{frame}

\begin{frame}
  \frametitle{Description Language: Example}
  \begin{figure}
   \centering
   \includegraphics[width=11cm]{top.png}
   \caption{Overview of metadata file for package trig in NASALib.}
  \end{figure}

\end{frame}

% Obtaining the pvslm implementation
\section{Obtaining the \cde{pvslm} Implementation}

\begin{frame}
  \frametitle{Agenda} 
  \tableofcontents[currentsection]
\end{frame}

\begin{frame}
   \frametitle{Obtaining \cde{pvslm}}

   The \cde{pvslm} tool can be installed automatically from the
   terminal in Unix-like systems by issuing the following command:
   
   \bigskip
   
   %
   \begin{small}
   \begin{tt}
     curl http://migueleci.github.io/pvslm/downloads/pvslm-conf.py 
      -o pvslm-install $\&\&$ chmod +x pvslm-install $\&\&$ 
      python ./pvslm-install
   \end{tt}
   \end{small}
   %

\end{frame}

% Available Commands
\section{Overview of Available Commands}

\begin{frame}
  \frametitle{Agenda} 
  \tableofcontents[currentsection]
\end{frame}

\begin{frame}
  \frametitle{Available Commands: Libraries}
  
   \begin{outeritemize}
   \item Adding a library source (i.e., command \cde{-a}) with a
    name, a short description, and a \cde{git} URL.
   \item Deleting a library source (i.e., command \cde{-d}) with
    the given name.
   \item Cloning a library source (i.e., command \cde{-c}) with the
    given name.
   \item Updating a library source (i.e., command \cde{-u}) with
    the given name.
   \item Removing the clone of a library source (i.e., command
    \cde{-r}) with the given name.
   \end{outeritemize}
  
\end{frame}

\begin{frame}
  \frametitle{Available Commands: Packages}
  
   \begin{outeritemize}
   \item Installing and updating a given package (i.e., command
    \cde{-i}) from a given library source, including {\em all} its
    dependencies.
  \item Updating a given package (i.e., command \cde{-u}) from a given
    library source, including {\em all} its dependencies.
  \item Deleting a given package (i.e., command \cde{-d}) from a given
    library source (local copy), including all packages that depend on
    it.
  \item Listing the contents (i.e., command \cde{-l}) from a given
    library source.
   \end{outeritemize}
  
\end{frame}

% Conclusion
\section{Conclusion}

\begin{frame}
  \frametitle{Agenda} 
  \tableofcontents[currentsection]
\end{frame}

\begin{frame}
  \frametitle{Conclusion}

  \begin{outeritemize}
  \item The \cde{pvslm} tool features support for different library
    sources, libraries with several theories, and dependencies among
    the theories (within the same library source).

  \item The tool is freely available for download and it is
    distributed under GNU's GPLv3 license.

  \item It uses a small footprint language for annotating libraries.
  \end{outeritemize}
\end{frame}

\begin{frame}{Useful Links}
  \begin{description}
  \item[\cde{pvslm}] \url{http://migueleci.github.io/pvslm}
  \item[]
  \item[\cde{NASALib}] \url{https://github.com/nasa/pvslib}
  \end{description}
\end{frame}
\begin{frame}

  \begin{center}
      {\LARGE \tbf{Thank you.}}
  \end{center}
\end{frame}

\end{document}

\begin{frame}
  \frametitle{The Problem}

\end{frame}

